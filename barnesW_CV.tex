%% start of file `template.tex'.
%% Copyright 2006-2013 Xavier Danaux (xdanaux@gmail.com).
%
% This work may be distributed and/or modified under the
% conditions of the LaTeX Project Public License version 1.3c,
% available at http://www.latex-project.org/lppl/.


\documentclass[10pt,a4paper,sans]{moderncv}        % possible options include font size ('10pt', '11pt' and '12pt'), paper size ('a4paper', 'letterpaper', 'a5paper', 'legalpaper', 'executivepaper' and 'landscape') and font family ('sans' and 'roman')

% moderncv themes
\moderncvstyle{banking}                             % style options are 'casual' (default), 'classic', 'oldstyle' and 'banking'
\moderncvcolor{black}                               % color options 'blue' (default), 'orange', 'green', 'red', 'purple', 'grey' and 'black'
%\renewcommand{\familydefault}{\sfdefault}         % to set the default font; use '\sfdefault' for the default sans serif font, '\rmdefault' for the default roman one, or any tex font name
%\nopagenumbers{}                                  % uncomment to suppress automatic page numbering for CVs longer than one page

% character encoding
%\usepackage[utf8]{inputenc}                       % if you are not using xelatex ou lualatex, replace by the encoding you are using
%\usepackage{CJKutf8}                              % if you need to use CJK to typeset your resume in Chinese, Japanese or Korean

% adjust the page margins
\usepackage[scale=0.75]{geometry}
%\setlength{\hintscolumnwidth}{3cm}                % if you want to change the width of the column with the dates
%\setlength{\makecvtitlenamewidth}{10cm}           % for the 'classic' style, if you want to force the width allocated to your name and avoid line breaks. be careful though, the length is normally calculated to avoid any overlap with your personal info; use this at your own typographical risks...

% personal data
\name{Will}{Barnes}
\title{CV}                               % optional, remove / comment the line if not wanted
\address{6100 Main Street MS-61}{Houston, TX 77005}{USA}% optional, remove / comment the line if not wanted; the "postcode city" and "country" arguments can be omitted or provided empty
\phone[mobile]{+1~(405)~308~-~0473}                   % optional, remove / comment the line if not wanted; the optional "type" of the phone can be "mobile" (default), "fixed" or "fax"
\email{Will.T.Barnes@Rice.edu}                               % optional, remove / comment the line if not wanted
\social[github]{wtbarnes}

% to show numerical labels in the bibliography (default is to show no labels); only useful if you make citations in your resume
%\makeatletter
%\renewcommand*{\bibliographyitemlabel}{\@biblabel{\arabic{enumiv}}}
%\makeatother
%\renewcommand*{\bibliographyitemlabel}{[\arabic{enumiv}]}% CONSIDER REPLACING THE ABOVE BY THIS

% bibliography with mutiple entries
%\usepackage{multibib}
%\newcites{book,misc}{{Books},{Others}}
%----------------------------------------------------------------------------------
%            content
%----------------------------------------------------------------------------------

\AfterPreamble{\hypersetup{
  pdfauthor={Will Barnes},
  pdftitle={CV for Will Barnes},
  pdfsubject={Detailed CV for Will Barnes},
  urlcolor=blue,
}}

\begin{document}
%\begin{CJK*}{UTF8}{gbsn}                          % to typeset your resume in Chinese using CJK
%-----       resume       ---------------------------------------------------------
\makecvtitle

%\section{Personal Information}
%\cvitem{Birthdate}{15 October 1990}
%\cvitem{Citizenship}{USA}
%
\section{Education}
\cventry{expected 2018}{Doctor of Philosophy in Physics}{Rice University}{Houston, TX USA}{}{}  % arguments 3 to 6 can be left empty
\cventry{2013--present}{Master of Science in Physics}{Rice University}{Houston, TX USA}{GPA: \textit{3.84/4.00}}{
\begin{itemize}
	\item Thesis:~\textit{Impulsive Heating in the Solar Atmosphere} (expected Fall 2015)
	\item Advisor:~Stephen Bradshaw, Ph.D.
\end{itemize}
}
\cventry{2009--2013}{Bachelor of Science in Astrophysics}{Baylor University}{Waco, TX USA}{GPA: \textit{3.89/4.00}}{Minors: Mathematics,~Great Texts of the Western Tradition
\begin{itemize}
	\item \textit{Magna Cum Laude}, Phi Beta Kappa
	\item University Honors Program
	\begin{itemize}
		\item Thesis:~\textit{Astrophysical Applications of Dusty Plasma Physics}
		\item Advisor:~Lorin Matthews, Ph.D.
	\end{itemize}
\end{itemize}
}
%
\section{Computing Skills}
\cvitem{Languages}{C/C++, Python}
\cvitem{Software}{git, IDL, \LaTeX, Mathematica, MATLAB, TORQUE}
\cvitem{Operating Systems}{Linux, Mac OS}
%
\section{Research Experience}
\cventry{2013--present}{Graduate Research Assistant}{Rice University}{Houston, TX}{Advisor: Stephen Bradshaw, Ph.D.}{
Research assistant in space physics division of the Department of Physics and Astronomy, Rice University. Research duties concentrated in computational solar physics. Teaching duties include, but are not limited to, a minimum of five semesters of leading lab sections of introductory physics.}
\cventry{June 2012--August 2012}{NSF REU Research Fellow}{CASPER, Baylor University}{Waco, TX}{}{
Accepted to National Science Foundation Research Experience for Undergraduates program in the Center for Astrophysics, Space Physics, and Engineering Research, Baylor University. Studied the effects of dust grain charging on aggregate size in a protoplanetary disk. Numerical work in extending kinetic model of grain growth to examine effect of disk location on grain charging.}
\cventry{June 2011--August 2011}{Summer Undergraduate Research Assistant}{Baylor University}{Waco, TX}{}{
Awarded a Summer Undergraduate Research in Physics (SURPh) grant from Department of Physics, Baylor University. Conducted research on anomalies in Saturn's F Ring by improving numerical models that simulate perturbed orbits of charged dust grains in a plasma environment.}
%
\section{Research Interests}
Broadly, my research interests are in solar physics, specifically in the numerical modeling of plasma dynamics in the solar corona. Specifically, I am interested in using hydrodynamic models to study nanoflare heating in the corona and how these modeled results can be compared to observations from instruments. Additionally, I am interested in using methods of non-negative matrix factorization to determine properties of nanoflare heating through analysis of observational results.
%
\section{Talks and Posters}
\cventry{21-23 July 2015}{Unversity of Cambridge}{Coronal Loop Workshop VII}{Cambridge, UK}{}{Poster title:~\textit{Effects of Ion Heating on Emission Measure of Coronal Loops in Active Region Cores}}
%
\cventry{26-30 April 2015}{American Astronomical Society}{Triennial Earth-Sun Summit}{Indianapolis, IN}{}{Poster title:~\href{http://adsabs.harvard.edu/abs/2015TESS....120306B}{\textit{Nonnegative Matrix Factorization as a Method for Studying Coronal Heating}}}
%
\cventry{18-22 March 2013}{Lunar and Planetary Science Institute}{44\textsuperscript{th} Annual Lunar and Planetary Science Conference}{The Woodlands, TX}{}{Poster title:~\href{http://adsabs.harvard.edu/abs/2013LPI....44.1897B}{\textit{Dust Grain Growth in a Protoplanetary Disk: Effects of Location on Charge and Size}}}
%
\cventry{14 September 2012}{Texas A\&M University}{Texas Undergraduate Astronomy Research Symposium}{College Station, TX}{}{Talk title:~\textit{Dust Grain Charging in a Protoplanetary Disk}}
%
\section{Honors and Awards}
\cvlistitem{Studentship Travel Award for 2015 SPD Annual Meeting, Solar Physics Division of the American Astronomical Society}
\cvlistitem{Dean's List, 7 of 8 semesters, Baylor University}
\cvlistitem{President's Gold Scholarship (GPA of at least 3.0, 12 semester hours), all semesters}
\cvlistitem{Gordon K. Teal Scholarship, 2 academic years}
\cvlistitem{Herbert D. Schwetman Scholarship, 2 academic years}
\cvlistitem{2013 URSA Scholars Week Outstanding Research Poster in Physics}
%
\section{Teaching Experience}
\cventry{Spring 2014, 2015}{Lab Teaching Assistant}{PHYS 102: Electricity and Magnetism}{}{}{Instructed lab sections of 40+ undergraduate students on topics including electrostatic interactions, magnetic induction, and basic circuits.}
\cventry{Fall 2014}{Lab Teaching Assistant}{PHYS 101: Mechanics}{}{}{Instructed lab sections of 40+ undergraduate students on topics including kinematics, collisions, and simple harmonic motion.}
%
\section{Societies and Associations}
\cventry{April 2009-May 2013}{National Honors Society}{Alpha Lambda Delta}{}{}{Completed 10 hours of service per semester.}
\cventry{September 2010-May 2013}{National Service Fraternity, Zeta Omega chapter}{Alpha Phi Omega}{}{}{Served as historian and treasurer. Completed 35 hours of service per semester. Managed finances for the organization. Organized a fundraiser.}
\cventry{April 2012-present}{National Physics Honors Society}{Sigma Pi Sigma}{}{}{Requirements for entry include being in upper-third of the class and completion of at least three semester of college course work in physics}
\cventry{September 2009-May 2013}{President}{Society of Physics Students}{}{}{As president, initiated rechartering of university chapter. Scheduled and presided over meetings. Organized end of the year luncheon and design and printing of t-shirts.}
%\cventry{September 2010-May 2013}{Discussion Group on Faith-Based Scholarship}{William Carey Crane Scholars}{}{}{Select group of students that met periodically throughout the semester to discuss topics in philosophy, theology and religion.}
%
\section{Employment Experience}
\cventry{January 2010-May 2013}{Office Assistant}{Department of Physics, Baylor University}{}{}{Assisted with examinations and attendance for class of 300. Helped with departmental events and mailing.}
%\section{Publications}
%\cvitem{1st Pub}{Description}
%\cvitem{2nd Pub}{Description}
%\cvitem{3rd Pub}{Description}

% Publications from a BibTeX file without multibib
%  for numerical labels: \renewcommand{\bibliographyitemlabel}{\@biblabel{\arabic{enumiv}}}% CONSIDER MERGING WITH PREAMBLE PART
%  to redefine the heading string ("Publications"): \renewcommand{\refname}{Articles}
\nocite{*}
\bibliographystyle{plain}
\bibliography{publications}                        % 'publications' is the name of a BibTeX file

% Publications from a BibTeX file using the multibib package
%\section{Publications}
%\nocitebook{book1,book2}
%\bibliographystylebook{plain}
%\bibliographybook{publications}                   % 'publications' is the name of a BibTeX file
%\nocitemisc{misc1,misc2,misc3}
%\bibliographystylemisc{plain}
%\bibliographymisc{publications}                   % 'publications' is the name of a BibTeX file

\end{document}


%% end of file `template.tex'.
